\documentclass[a4paper]{article}

\usepackage[english]{babel}
\usepackage[utf8]{inputenc}
\usepackage{amsmath}
\usepackage{graphicx}
\usepackage[colorinlistoftodos]{todonotes}
\usepackage[numbers]{natbib}

\title{How biased is the Henry Buckley Collection of Planktonic Foraminifera?}

\author{Marina C. Rillo, Michal Kučera, Thomas H. G. Ezard, \\ Andy Purvis \& C. Giles Miller}

\date{\today}

\begin{document}
\maketitle

%\begin{abstract}
%Enter a short summary here. What topic do you want to investigate and why? What experiment did you perform? What were your main results and conclusion?
%\end{abstract}


%----------------------------------------------------------------------------------------

\section{Introduction}
\label{sec:introduction}

The original deposits, from which the Buckley collection was created, are still stored at the NHMUK Ocean-Bottom Deposits (OBD) Collection. 
There is no record of how Buckley picked the specimens to amass his collection. However, since he personally carried out the sample processing, isolation of foraminiferal specimens and their identification, all biases in his collection are likely to be systematic. The two main potential sources of bias are taxonomic bias (systematic misidentification or incomplete representation of the assemblage) and size bias (bias towards larger specimens or larger lineages). The presence of such bias could significantly affect trait distribution and variance. Therefore, the recognition of the biases in the Buckley Collection is crucial to use it to study morphological variation. The bias information assessed in the course of my research in Bremen will be placed together with the collection dataset on the open online NHMUK Data Portal and will be valuable for other researchers that want to use this collection in the future.



%----------------------------------------------------------------------------------------

\section{Methods}
\label{sec:methods}

	\subsection{Sampling and picking}
	
The NHMUK Ocean-Bottom Deposits (OBD) Collection houses most of the sea-bottom samples used by Henry Buckley to amass his planktonic foraminifera collection. These samples were collected by historical marine expeditions spanning mostly from the late 1800s until mid 1900s. 
We re-sampled ten OBD samples used by Buckley in his collection (Table \ref{table:samples}). We focussed on core-top samples which encompass different oceans, latitudes and marine expeditions. The final choice of the ten OBD samples, however, could not be completely randomized since it depended on the amount of sediment available for re-sampling as well as us being able to identify the bulk sample in the NHMUK off-site storage facility.
Once we defined the ten samples, we took roughly half of the amount available in the OBD jars and tubes (see Table 1\ref{table:samples} for sampled masses). Each of these samples was further split into two equal parts, leaving an archive sample and a sample to be processed. The processing of the samples consisted of weighting each of them, then wet-washing over a 63$\mu$m sieve and drying in the 60C oven. The samples were further dry-sieved over a 150$\mu$m sieve and the fraction bigger than 150$\mu$m was further split with a microsplitter to produce a sample containing around 300 specimens. All specimens in these final splits were picked and identified (\textbf{supp info with species table for each sample}). In total 2837 specimens were picked, identified and mounted on slides. These slides are now part of the Henry Buckley Collection at the Micropalaeotology Section NHMUK and can be used as a type-collection covering intraspecific morphological variability across each species' biogeographical range. 

	

\begin{table}


\caption{\label{table:samples} Sediments re-sampled for bias analysis}

\centering
%\singlespacing
\setlength{\tabcolsep}{0.2cm}

\makebox[\linewidth]{ % put table in middle of the page
\footnotesize
\begin{tabular}{ r c r r c r c}
\hline

\textbf{OBD IRN} & \textbf{Ocean} & \textbf{Latitude} & \textbf{Longitude} & \textbf{Sea Depth (m)} & \textbf{Mass (g)} & Total no. ind.\\ 

\hline

% sample 5
32657 & Indian & -50.01667 & 123.06667 & -3976 & grams & 318\\
% sample 10
38482 & Indian & -40.45000 & 49.81667 & -3780 & grams & 177\\
% sample 20
36053 & Indian & -26.93667 & 111.18167 & -3350 & grams & 279\\
% sample 25
34991 & Atlantic & -21.25000 & -14.03333 & -3740 & grams & 265\\
% sample 27
34671 & Indian &  -19.56667 & 64.63333 & -2708 & grams & 376\\
% sample 31
34993 & Pacific & -15.65000 & -179.06250 & -2519 & grams & 300\\
% sample 44
37148 & Indian & -7.59167 & 61.48333 & -3507 & grams & 305\\
% sample 46
33668 & Pacific & -0.70000 & 147.00000 & -2213 & grams & 331\\ 
% sample 55
33286 & Atlantic & 24.33333 & -24.46667 & -5153 & grams & 260\\ 
% sample 66
14609 & Arctic & 85.25000 & -167.90000 & -1774 & grams & 226\\ 

\hline

 & &  &  &  &  & 2837\\ 

\hline %inserts single line

\end{tabular}

} % makebox

\end{table}



	\subsection{Size measurements}
	
To obtain the size distribution from the bulk sediments we manually mounted specimens on slides. The shell position on the slide will correspond to the shell position Buckley established for each lineage. These slides will were imaged and the foraminiferal shell size was measured using in NOC Southampton. Brombacher et al. (in preparation) quantified the reproducibility of shell size measurements and concluded that it is highly consistent with remounting the slides.


	\subsection{Data Analysis}
Taxonomic bias was assessed by comparing for each sample the species identified by us and the ones present in the Buckley Collection. Dissimilarity 
	
% If taxonomic bias is present there would be a mismatch between the lineages found in Buckley’s Collection and the ones found in my sample. If Buckley found lineages that were not observed by me, this might indicate he misidentified them. In this case, I will carefully re-analyse the high-resolution images of the potentially misidentified lineage in his collection to check whether that lineage was systematically misidentified. When the misidentification of a lineage is consistent throughout the collection, we will only use my samples’ specimens for the analysis. If, however, 
	
	
Size bias can be detected as a bias towards larger specimens or larger lineages. The latter has the same effect as an incomplete representation of the assemblage, as it would mean that Buckley only identified a sub-sample (large lineages) of the full assemblage. Size bias towards larger individuals will be assessed by comparing the shell size distribution obtained from my re-sampling of the sediments and the one obtained from the Buckley collection. Shell size distributions were compared using statistical test. 

% If I find evidence of a specimen size bias in the collection, I will clarify for each lineage (i) how do the size distributions differ (in their average, their variance or in both)? (ii) Do the size distributions of a lineage differ in a systematic way (i.e. is the size bias consistent throughout the sediments where the lineage is found)? The answer to these questions will guide me on how to adjust my dataset to the biases found.

Size distributions already have an artificial cut-off dictated by the mesh size of the sieve used when processing the bulk sediment. This artificial cut-off influences each lineage differently, because depending on the lineage’s average shell size the cut-off will eliminate a different portion of its population. % If the size bias in the Buckley collection is systematic among the populations within a lineage, we could parametrize it as an additional artificial cut-off. 

% I will determine the populational maximum shell size distribution based on Schmidt et al. (2004) methodology. They used the size value that separates the largest 5\% of shells (i.e. the 95th centile) and showed that this value is more correlated to the mean and median of the size distribution than the maximum value itself, because the maximum value is more dependent on random sampling (Schmidt et al. 2004). If the size bias is not systematic within lineages of the collection, then the adjustment of the collection morphometric dataset will depend on how the size distributions differ.  It might also be that not all populations of a lineage show size bias, allowing us to still use some populations for the analyses.


%----------------------------------------------------------------------------------------

\section{Results}


	\subsection{Taxonomic bias} we expected that Buckley's species richness for each sample would be a subsample (i.e. nested) of the species richness found by us. A incomplete representation of the assemblage.

	\subsection{Size bias}





%----------------------------------------------------------------------------------------

\section{Discussion}

\begin{itemize}
\item Rarefaction curve MARGO
\item Similarity among neighbouring samples MARGO
\item bias of rare species
\end{itemize}

% If the lineages Buckley identified are a subsample of the lineages recognized by me, this indicates an incomplete representation of the assemblage, meaning that the sediment’s planktonic foraminiferal richness is not well represented in the Buckley collection. This type of taxonomic bias is critical for the use of the collection in analyses of planktonic foraminiferal communities and species coexistence, which will probably be a later section of my PhD. However, the current proposed project focuses on within lineage variation (intraspecific morphological variation, species’ abundance and genetic diversity), therefore this bias will not directly affect our analyses.



%----------------------------------------------------------------------------------------

\section{Conclusion}




%----------------------------------------------------------------------------------------

\label{Bibliography}
\bibliographystyle{abbrvnat} % Use the agsm, abbrvnat, unsrtnat BibTeX style for formatting the Bibliography
\bibliography{bibliography.bib} % The references (bibliography) information are stored in the file named "Bibliography.bib"



\end{document}  

